% CU CS5525
% Fall 2012
% Python Compiler
%
% proposal.tex
% Semester Project Proposal
%
% Repository:
%    https://github.com/asayler/CU-CS5525-PythonCompiler
%
% By :
%    Anne Gatchell
%       http://annegatchell.com/
%    Andy Sayler
%       http://www.andysayler.com
%    Michael (Mike) Vitousek
%       http://csel.cs.colorado.edu/~mivi2269/

\documentclass[11pt]{article}

\usepackage[text={6.5in, 9in}, centering]{geometry}
\usepackage{graphicx}
\usepackage{url}
\usepackage{listings}
\usepackage{hyperref}

\bibliographystyle{plain}

\hypersetup{
    colorlinks,
    citecolor=black,
    filecolor=black,
    linkcolor=black,
    urlcolor=black
}

\newenvironment{packed_enum}{
\begin{enumerate}
  \setlength{\itemsep}{1pt}
  \setlength{\parskip}{0pt}
  \setlength{\parsep}{0pt}
}{\end{enumerate}}

\newenvironment{packed_item}{
\begin{itemize}
  \setlength{\itemsep}{1pt}
  \setlength{\parskip}{0pt}
  \setlength{\parsep}{0pt}
}{\end{itemize}}

\begin{document}

\title{
  Building a LLVM Python Compiler
}

\author{
  Anne Gatchell \and Andrew Sayler \and Michael Vitousek\\
  University of Colorado\\
  \texttt{first.last@colorado.edu}
}

\date{\today}

\maketitle

\newpage

\section{Problem Statement}

While the existing HW6, P3-compliant compiler supports 32-bit,
x86 code generation, it does not support generating code for the variety
of non-x86 architectures and assembly languages common today (x64, ARM,
etc). While we could remedy
this deficiency by adding additional native code generation for non-x86
architectures directly to our compiler, this approach would duplicate a
wide range of existing effort, force us to work in assembly languages
with which we are not experts, and would require continued maintain to
support new architectures and assembly languages as they arise.

Instead, we aim to leverage the existing work done by the LLVM project to
convert our HW6 compiler from an x86 targeted compiler to
an LLVM targeted compiler. We will maintain the current x86 targeting
and add LLVM intermediate form as an additional targeting option. In
this way, we will be able to compare our natively generated x86 code
with the LLVM-assembler generated x86 code. We will also be able to
experiment with compiling our LLVM intermediate form files to assembly
languages other than x86 such as x64 and ARM.

LLVM is quickly becoming the de facto standard target for most modern
compilers and high level languages.
It provides mutli-platform support, multi-runtype support
(compiled, interpreted, JIT compiled, etc), and access to a
range of existing optimization tools and techniques. Through this
project, we hope to become
familiar with the LLVM intermediate form and LLVM system architecture
while gaining insight into the benefits of building an LLVM-targeted
compiler. 

\section{Approach}

% Mention something about the various steps involved. I.e. add support
% for SSA form, add LLVM instruction nodes, add LLVM instruction
% selection, convert Makefile to support LLVM builds, investigate
% compiling runtime functions to LLVM and linking in LLVM, comparing
% LLVM performance vs x86 performance, playing with non-x86 LLVM
% targets, playing with various LLVM runtypes, etc

\section{Design}

% Mention keeping most of the re-factored HW6, p3 design, but changing
% instruction selection, ability to forgo register allocation, need to
% add SSA form conversion pass, etc

\section{Deliverables}

% 1. Basic SSA form support with native x86 code
% 2. Basic LLVM instruction selection support
% 3. Pass all current test cases using LLVM
% 4. (if time) Compare LLVM performance to native x86 performance
% 5. (if time) Compare LLVM runtypes (interpreted, compiled, etc)
% 6. (if time) Test LLVM implementation on x64 and ARM architectures
% 7. (if time) Look into adding LLVM optimization for typing, JIT, etc

\nocite{*}
\bibliography{refs}

% If anyone finds one, we could use one or more good SSA reference in
% the bibtek file. Also, the more LLVM reference, tutorials, etc the
% better. Maybe the original LLVM papers? LLVM ``Hello World''
% examples? Reports on other systems that use LLVM (Apple, Obj-C, etc)?

\end{document}
